\documentclass[12pt]{article}
\usepackage{amsfonts}
\usepackage{amsmath}
\renewcommand\labelenumi{(\roman{enumi})}
\renewcommand\theenumi\labelenumi
\begin{document}
\title{Solutions and Notes to Michael Spivak's \emph{Calculus}}
\author{Community}
\date{}
\maketitle
\cleardoublepage

\section{Basic Properties of Numbers}
\subsection{}
Prove the following.
\subsubsection{}
If $ax=a$ for some number $ a\neq 0$, then $x=1$.
\begin{equation*}
\begin{split}
ax&=a\\ 
x&=\frac{a}{a}\\
x&=1
\end{split}
\end{equation*}
\subsubsection{}
$x^2-y^2=(x-y)(x+y)$
\begin{equation*}
\begin{split}
x^2 - y^2 &= x^2 + xy - xy - y^2\\ 
x^2 - y^2 &= x^2 - y^2
\end{split}
\end{equation*}

\subsection{}
What is wrong with the following "proof"? Let $x=y$.
\begin{equation*}
\begin{split}
(x+y)(x-y) &= y(x-y)\\ 
x + y &= y
\end{split}
\end{equation*}
We divide by $(x-y)$, which given $x=y$ equals $0$. We cannot divide by zero.

\end{document}